\section{Introduction}
\label{introduction}

The phenomenon of technical debt has matured in recent years with numerous
scientific experiments conducted for its identification, measurement and
management. Initially coined in 1993 \cite{Cunningham1993}, it defined the
concept of not quite right code that provided gains in the short term. Thus
technical debt might be useful in achieving immediate deadlines with possible
software quality sacrifices that might be negative for future work. Negative
consequences in the long term could see software complexity growth, the
prevalence of bugs and defects, decreased team productivity and augmented work
effort, leading to increased costs of development, infrastructure and
management. The phenomenon affects daily work of developers, project managers
and stakeholders of the business.

The definition of technical debt has been updated since then, spanning not just
activities related to code implementation but the entire software development
environment. The scientific community identified multiple types of debts
\cite{Li2015}, each with its advantages and consequences if not handled
accordingly. Although industry practitioners are aware of its presence
\cite{Codabux2013} \cite{Lim2012}, there is no standard way of measuring the
current and future impact of technical debt on development and costs of the
team. Additionally, due to lack of vocabulary and complexities of the
phenomenon, developers find it difficult to convey their concerns to project
stakeholders \cite{Kruchten2012}.

The primary research objective is to analyse work effort of feature
implementations within software projects with the purpose of identifying extra
work in the context of technical debt. An additional objective is to understand
how the measure of technical debt varies with software evolution and what types
of features incur or reduce its presence. The research questions have been
organised using the Goal-Question-Metric approach \cite{VanSolingen2002}:
\begin{itemize}
	\item \textbf{RQ1}. Can technical debt be measured in the context of work
	      effort?
	      \begin{itemize}
		      \item \textbf{RQ1.1}: What was the difference (delta) in the
		            estimated work effort for a feature and the practical work
		            effort?
		      \item \textbf{RQ1.2}: What was the level of technical debt across
		            feature changesets at the time of implementation?
		      \item \textbf{RQ1.3}: How does the work effort delta vary with the
		            magnitude of technical debt?
	      \end{itemize}
	\item \textbf{RQ2}: What are the development patterns surrounding feature
	      lifecycle?
	      \begin{itemize}
		      \item \textbf{RQ2.1}: At what checkpoints in feature development
		            is technical debt reduction (refactoring) most prominent?
		      \item \textbf{RQ2.2}: What type of work items incur the most
		            technical debt?
	      \end{itemize}
\end{itemize}

The main questions RQ1 and RQ2 will be tackled by answering their following
subquestions. Additionally, the following metrics will be computed and analysed:
\begin{itemize}
	\item The \textbf{practical work effort} spent on feature implementation
	      will be calculated by leveraging data from the team's issue tracker and
	      version control repository.
	\item The level of \textbf{technical debt} in the system at the time of
	      implementation will be measured in terms of code smells and bad coding
	      practices.
	\item The \textbf{extra effort} spent on feature implementation will be
	      derived as the difference between the estimated and practical effort
	      measurements.
\end{itemize}

As data candidates, the study will use both open-source and enterprise projects
in order to understand the differences between the two worlds in the context of
technical debt work effort. The candidates will be selected according to various
criteria items such as feature development model, type of version control
system, availability of estimated work effort metadata and suitability of code
quality tools. 

Our long term goal is to provide a decision making framework for the management of
technical debt in software projects. This framework may provide essential
historical information which might be useful for all roles within the software
development environment. For example, individual developers could find out how
much extra effort was spent in areas with a high number of technical debt items.
Architects could identify architectural bottlenecks quickly whilst project
managers would assign work based on empirical evidence gathered from past work
effort. As of January 2018, the study is in initial stages and work is currently
underway for identifying project candidates and extracting features with
comparable costs.

In Section \ref{related-work}, we contrast our study with others and how we
build upon current research. Section \ref{work-plan} highlights the proposed
work followed by its limitations in Section \ref{limitations}.  

\section{Related Work}
\label{related-work}

\section{Work Plan}
\label{work-plan}

\section{Limitations}
\label{limitations}

\section{Conclusion}
\label{conclusion}

To conclude, technical debt is a phenomenon difficult to measure accurately and
assess potential development and business costs. Therefore, understanding it
from the perspective of developers is vital as they are first-hand involved in
the implementation of new features. Any extra work spent as a result of previous
incurred debt increases business costs. If too much debt accrues over the
lifetime of a project, the entire project may be brought to a stand-still.

As a result, this study will try to understand technical debt from a development
work effort perspective. It will, possibly, shed light on the types of features
that take a lot of working hours to complete in correlation with the level of
technical debt at the time of the implementation.

Through measurement and analysis of these two metrics and their relationship,
developers may discover bottlenecks in productivity, managers may allocate
appropriate resources for refactoring activities, and business stakeholders may
become more aware of development concerns. Additionally, it will be interesting
to find the discrepancy between open-source and proprietary software projects.

To the best of our knowledge, this type of study had not been conducted before
from the perspective of work effort at such a low granularity level and may
provide valuable insights into the development patterns of software engineering.

% \end{document}  % This is where a 'short' article might terminate


% \begin{acks}
%   The authors would like to thank Dr. Yuhua Li for providing the
%   MATLAB code of the \textit{BEPS} method.

%   The authors would also like to thank the anonymous referees for
%   their valuable comments and helpful suggestions.
% \end{acks}
