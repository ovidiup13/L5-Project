\section{Introduction}
\label{introduction}

The metaphor of technical debt was first proposed by \cite{Cunningham1993}, in
reference to the temporary sacrifice of code quality in exchange for the faster
delivery of features. Thus, incurring technical debt might be useful to achieve
immediate deadlines but may simultaneously have negative consequences for future
work, such as software complexity growth, the prevalence of bugs and defects,
decreased team productivity and augmented work effort, leading to increased
costs of development, infrastructure and management. The impact of this
technical debt may affect the daily work of developers, project managers,
product owners and other stakeholders of the business.

Although initially intended as a metaphor, considerable research has since been
undertaken into the identification \cite{Lim2012}, measurement \cite{Fontana2016} and
management \cite{Codabux2013} of technical debt.  Research has also expanded the
scope of the term, such that technical debt spans not just activities related to
code implementation but the entire software development environment
\cite{Power2013}. The research community identified multiple types of debts
\cite{Li2015}, each with its advantages and consequences if not handled
accordingly. Although industry practitioners are aware of its presence
\cite{Codabux2013} \cite{Lim2012}, there is no standard way of measuring the
current and future impact of technical debt on development and costs of the
team. Additionally, due to lack of vocabulary and complexities of the
phenomenon, developers find it difficult to convey their concerns to project
stakeholders \cite{Kruchten2012}.

For example, code smells are an example of technical debt items that may
increase future development effort and maintenance costs \cite{Fowler1999}.
Industry case studies have shown that such code violations are more likely to be
addressed if made visible by developers within the project's issue tracker
\cite{Lim2012}.  However, from the perspective of a project manager, it is
essential to understand how much effort the team puts into technical debt
reduction activities and if there is an associated business value.  A key
concern, therefore, is the ability to measure and make rationale cost-benefit
analyses of the impact of technical debt on productivity within a project,
capturing both the principal debt incurred and consequent interest, as poor code
quality accumulates.

The primary research objective of our work is to identify the accumulated
additional work effort caused by the presence of technical debt in a code base,
through an empirical analysis of work effort of feature implementations within
software projects.  Our intention is to determine whether a decision to delay
repaying technical debt through refactoring work creates additional work effort
in the longer term to repay that debt. I.e. does technical debt really incur
interest to the principal, and if so, in what way?  An additional objective is
to understand how the measure of technical debt varies with software evolution
and what types of features incur or reduce its presence.  The research questions
for our work, organised using the Goal-Question-Metric approach
\cite{VanSolingen2002}, are as follows:

\begin{itemize}
	\item \textbf{RQ1}. Can technical debt be measured in the context of work
	      effort?

	\item \textbf{RQ1.1}. How does the work effort cost of feature implementation
              vary with the magnitude of technical debt? 

        \item \textbf{RQ1.2}. Does technical debt incur interest in terms of increasing
              work effort cost of feature implementation as removal of technical debt 
              items is delayed?

	\item \textbf{RQ2}: What are the development patterns surrounding feature
	      lifecycle?
	      \begin{itemize}
		      \item \textbf{RQ2.1}: At what checkpoints in feature development
		            is technical debt reduction (refactoring) most prominent?
		      \item \textbf{RQ2.2}: What type of work items incur the most
		            technical debt?
	      \end{itemize}
\end{itemize}

Our intention is to adopt an empirical approach to the identification of the
impact of technical debt on the implementation of future features in existing
software code bases.  To do this, we will calculate the level of
\textbf{technical debt} in the system for each change-set recorded in a
project's version control repository.  We will also record calculate the
\textbf{practical work effort} spent on feature implementation linking data from
a project's issue tracker and version control repository and thus estimating the
effort a developer spent on completing a feature.  The impact of technical debt
on work effort will then be calculated by comparing the practical work effort
with both the estimated effort for an issue.

As data candidates, the study will use both open-source and enterprise projects
in order to understand the differences between the two worlds in the context of
technical debt work effort. A large Fortune 500 institution has agreed to
collaborate with us by granting access to their software development environment
data. The candidates will be selected according to various criteria items such
as feature development model, type of version control system, availability of
estimated work effort metadata and suitability of code quality tools. 

Our long-term goal is to provide a decision-making framework for the management
of technical debt in software projects. This framework may provide essential
historical information which might be useful for all roles within the software
development environment. For example, individual developers could find out how
much extra effort was spent in areas with a high number of technical debt items.
Architects could identify architectural bottlenecks quickly whilst project
managers would assign work based on empirical evidence gathered from past work
effort. As of January 2018, the study is in its initial stages.

The rest of this paper is structured as follows.  In Section \ref{related-work},
we contrast our study with others and how we build upon current research.
Section \ref{work-plan} highlights the proposed work followed by its limitations
in Section \ref{limitations} and Section \ref{conclusions} summarises the
proposal.

\section{Related Work}
\label{related-work}

There are a number of studies which have investigated the impact of technical
debt items on the costs of software projects.  This section reviews this work
and relates the existing state of the art to our proposed experiment to
empirically measure the work effort interest accumulated by technical debt.

Olbrich et al. \cite{Olbrich2009} studied the impact of two code smells, God
class and Shotgun Surgery, on change-proneness of class entities and size of
changes within two open source systems, Apache Lucene and Apache Xerces. They
found that classes containing either smell are more change prone than other,
non-smelly, classes. In a similar study, Khomh et al. \cite{Khomh2009}
empirically analysed the impact of 29 code smells on the change-sets of 9
releases of two open source systems. They had confirmed the results of the
previous studies that code smells increase the number of changes that software
undergoes during its evolution. Additionally, they had found that classes
containing more than one code smell are more change prone than other classes.
Charalampidou et al. \cite{Charalampidou2017} introduced a study that assessed
the interest probability of code smells, which is the probability of a code
smell to introduce extra changes in future development. Interest probability was
calculated by counting the frequency of each code smell and how it correlated
with the change-proneness of the module where it resides. The smells studied
were Long Method, Conditional Complexity and Duplicated Code. The results showed
that code duplication had the highest interest probability due to the number of
changes required to maintain future development. Additionally, high cyclomatic
method complexity increased the number of changes. Fontana et al.
\cite{Fontana2012} studied what was the impact of removing code smells on code
quality metrics such as cohesion, coupling and complexity and, which smell
incurred the most debt. They applied refactoring activities for each smell and
the metrics were re-evaluated. The results showed that refactoring of one code
smell might provide benefits for some metric qualities but may negatively impact
others.

The authors studied the impact of code smells and their removal on the software
quality metrics output by code quality tools. Although their research has
increased awareness of technical debt, our proposed work aims to focus on what
impact code smells have on feature implementation by identifying code smells
using automated tools. Additional studies have looked at quantifying
implementation cost in the context of technical debt.

Singh et al. \cite{Singh2014}, calculated interest payments by monitoring
development effort and code comprehension. They monitored time spent by
developers in classes with known technical debt items and implemented a tool
within the Integrated Development Environment of developers to gather
information on class visits and development session times. The interest was
quantified as the difference between time practically spent in classes and the
ideal time. However, the study was conducted with the input of only one
developer over a period of 9 months. Additionally, estimating the ideal time
spent on development is a challenging task due to social and personal factors
such as level of project knowledge, environment familiarity and programming
language preference.

Gomes et al. \cite{Gomes2011} studied the correlation between software
evolution, defect occurrence and work effort deviation at the release level. The
authors extracted data from documentation sources such as test plans, project
plan, weekly reports, project source code, and emails. Using this data, they
could derive important team information on change-sets, effort, quality, test
and size of the system. They measured extra work effort by subtracting the
estimated work time and total practical work time. Although information at the
release level offers project managers an idea of work effort deviation, it does
not show at a granular level what defects slow down development of a new feature
and where the team should focus their refactoring activities.

The initial study by Gomes et al. \cite{Gomes2011} provided a good start to
identifying work effort deviation. However, it only analysed major releases of a
system and did not provide drill-down information on iterations and code commits
on which features and code smells had the most effect on work effort. This is
essential information for developers and project managers to prioritise
refactoring activities in work iterations, especially in an Agile environment
where responding to change is critical \cite{agile-manifesto}.

\section{Experimental Design}
\label{work-plan}

This study focuses on finding the relation between technical debt and work
effort. This will yield more information for developers on time management, for
project managers on refactoring investment and for stakeholders on understanding
a technical concept concerning development cost.

The approach of this study will consist of the following steps:
\begin{enumerate}
	\item \textit{Identify appropriate data candidates for this study}.\\
	      These are potential software projects that are suitable for study.
	      At best, they should have multiple developers contributing to the
	      project, a medium to large codebase with a good amount of historical
	      data and an associated issue tracking software. Ideally, the team
	      would have integrated a continuous code quality tool for tracking
	      code smells throughout software evolution that would help with quick
	      and automatic identification of present and historical code issues.

	      The study would also benefit from a mixture of open-source and
	      enterprise software to contrast the differences between the two
	      environments and, possibly, arrive at general conclusions that may
	      contribute to both worlds.

	\item \textit{Identify suitable work items from issue tracker}.\\
	      In this step, the purpose is to understand significant events in the
	      evolution of the data candidates. Ideally, events were tracked in the
	      form of tickets with attached metadata, such as:
	      \begin{itemize}
		      \item \textit{Priority} will give a sense of importance to the
		            work item. Finding development patterns based on this
		            field will be unique. Will developers take more time to
		            design and implement an important work item?
		            Alternatively, are they pressured to deliver and thus
		            introduce more debt in the system?
		      \item \textit{Estimated Work Effort} is one of the most
		            important fields for calculation of extra work effort.
		            This provides the theoretical work time on which to
		            compare the practical work time measured in this
		            experiment. It may be in the form or work hours or story
		            points.
		      \item The opening and closing \textit{timestamps} of a ticket
		            may offer valuable information for measuring the
		            practical work effort in hours, if developers change the
		            status of tickets as development progresses.
	      \end{itemize}

	      Unfortunately this information is not always available. The most
	      important field is the estimated work effort value. Without it, the
	      extra work is difficult to identify. As a way to mitigate this
	      issue, we will not select feature tickets which do not contain this
	      field.

	\item \textit{Identify version control checkpoints}.\\
	      Completed work items can be tracked in the version control
	      repository of the project. Identifying checkpoints will aid in
	      understanding the amount of effort put into the work item by the
	      team and how it diverges from the initial estimation.

	      Ideally, the team should have links between revisions of the
	      codebase and the work item in the issue tracker. This would make it
	      easier to find the associated checkpoints.

	\item \textit{Measure the amount of work effort for each work item}.\\
	      The purpose is to understand how many practical changes a work item
	      has induced over its lifetime. This could be done in two ways: at
	      the code and issue tracker levels.

	      At the code level, it is possible to understand the level of work
	      effort involved by aggregating the number of changes a work item has
	      suffered. A change-set consists of the number of lines of code
	      added, deleted and modified. Granularity can be at the pull request,
	      commit, class and method level. Version control systems such as Git
	      provide features for retrieving change-sets between revisions.

	      However, identifying work effort from change-sets is a challenging
	      task. It is difficult to quantify in working hours since many
	      changes may be generated automatically by modern refactoring tools
	      in the integrated development environment. An alternative solution
	      is to compare the timestamps between the first and the last commit.
	      The temporal difference might provide a practical estimate of work
	      effort to resolve the issue. Unfortunately, this case only works
	      when a developer works on a single issue at a time.

	      At the ticket level, one can understand the amount of work effort
	      realised by a team member. Ideally, the team enforces developers to
	      log their time spent designing and implementing a feature. However,
	      that is not always the case. Alternatively, it would be interesting
	      to retrieve the timestamp of ticket events, such as the opening and
	      closing of an issue. Unfortunately, this might not give an
	      approximate time of work since:
	      \begin{itemize}
		      \item The team does not respect the opening and closing of a
		            ticket time according to their development patterns. For
		            example, a developer might start work on an issue before
		            marking it as ``In Progress'' and thus introduce a margin
		            of error.
		      \item Tickets might remain open for a long period of time,
		            while features are implemented in relatively quickly.
		      \item There are differences between enterprise and open-source
		            software. For example, developers might work in the
		            timeframe of 9 AM to 6 PM in an enterprise while in
		            open-source they are free to work at any time of the
		            day. For example, in an extreme case, a developer marks
		            a ticket as ``In Progress'' before the end of the
		            working day, and resume working the following morning.
		            In this case, our estimation of approximately 15 hours
		            of development time for this work item would be
		            incorrect.
	      \end{itemize}

	      For this study, the code level technique will be implemented. The
	      work effort from issue tracking will be implemented in a parallel
	      study. It will be interesting to gather results from both methods
	      and see how they correlate. Additionally, the two result sets may
	      complement one another and provide an overall effort metric.

	\item \textit{Measure technical debt items}.\\
	      The scope of this step is to identify code violations within
	      change-sets. For each work item implemented there will be a set of
	      associated modules, classes and methods affected. Historical code
	      smells can be identified and tracked within the evolution of
	      change-sets using a continuous code quality tool.

	      Ideally, the team would have the code quality tool integrated into
	      their continuous integration environment. If so, then historical
	      code smell data could be leveraged by retrieving it through an API.
	      Alternatively, a code quality tool will be used to analyse the
	      version control checkpoints identified and find code smells that may
	      have an impact on change-sets of a feature.

	\item \textit{Analysis and discussion of results}.\\
	      The three data sources can be linked together once all steps are
	      fulfilled. Extra work can be correlated to issue tracking
	      information and code quality at the time of development. Technical
	      debt could be classified by the type, priority, code smells and
	      assignee.
\end{enumerate}

Unfortunately, the proposed work is not as straightforward. There are many
complexities of the work environment which cannot be considered from the three
data sources. Therefore, we will make some assumptions to simplify the process:
\begin{itemize}
	\item The team uses Git for version control and follows the ``Pull
	      Request'' model for implementing changes.
	\item Only one developer is assigned to an issue.
	\item A developer works on maximum one issue at a time.
	\item The team uses an issue tracker consistently. Developers change the
	      status of the ticket according to their current development
	      progress. For example, if a team member got started on a work item,
	      she would set the ticket status to ``In Progress''. Respectively,
	      she would mark the ticket as ``Closed'' if development ceased, her
	      changes were reviewed and integrated into the main branch.
	\item The team estimates the theoretical amount of work necessary to
	      implement a feature or fix a bug. This measure is attached to each
	      work item selected.
	\item Development time is between 9 AM to 6 PM UTC. Only work items with
	      opening and closing status between these two values are taken into
	      consideration. For example, if a developer starts work on a feature
	      on Monday 5 PM and finishes it by Tuesday 1 PM, then development
	      time will be considered to be 4 hours (1 hour Monday, 3 hours
	      Tuesday). This assumption will simplify the calculation of
	      reasonable work hours for work items.
\end{itemize}

These assumptions will reduce the amount of real-world complexity in the
experiment and allow for simple validation of the results. For enterprise
projects, on-premise interviews with teams may help validate the results.

\section{Limitations}
\label{limitations}

The experimental design reported in this paper is a work in progress, although
we already recognise limitations to the approach. For example, there are ten
types of debt, as identified in a mapping study by Li et al.  \cite{Li2015}. In
this work, we only aim to examine code and design debt, as they are the most
prominent and that developers and project managers have to deal on a daily
basis. The primary challenge is measuring the implementation interest that
accrues over the lifetime of a project. Additionally, with the introduction of
assumptions to limit the complexity of the study, we have identified the
following threats to the validity of our future results:

\begin{itemize}
	\item Data candidates may not have sufficient information related to the
	      estimation of work effort. We believe this is specifically true for
	      open-source projects, implemented by international teams with
	      contributions from many developers.
	\item Work effort measurement is a difficult challenge and was deemed
	      unmeasurable by Martin Fowler \cite{CannotMeasureProductivity}. We
	      consider work effort measurement as the time taken to deliver the
	      requirements, assuming that all requirements of a system in
	      development brings an associated business value. Restricting
	      development time to an interval and discarding multi-developer work
	      items may reduce the data set considerably.
	\item Code quality tools may not detect potential technical debt items.
	\item The assumptions made in the previous section may not reflect the
	      real work environment. For instance, developers may be working
	      overtime if under the pressure of a release schedule.
	\item There are many types and complexities of technical debt which were
	      not included. Requirements, architectural, build, infrastructure and
	      test have an influence on the amount of effort for finalising a
	      feature.
\end{itemize}

Although there are risks associated with measurements, care will be taken to
consider all possibilities when validating calculation of measurements. This is
especially true in the case of quantifying work effort.

\section{Conclusion}
\label{conclusions}

To conclude, technical debt is a phenomenon difficult to measure accurately and
assess potential development and business costs. Therefore, understanding it
from the perspective of developers is vital as they are first-hand involved in
the implementation of new features. Any extra work spent as a result of
previously incurred debt increases business costs. If too much debt accrues over
the lifetime of a project, the entire project may be brought to a stand-still.

As a result, this study will try to understand technical debt from a development
work effort perspective. It will, possibly, shed light on the types of features
that take a lot of working hours to complete in correlation with the level of
technical debt at the time of the implementation. The first step of the study
will identify project candidates from the open-source and enterprise worlds
along with historical feature tickets and their associated implementations. The
next step will involve the measurement of work effort of these features through
analysing version control and issue tracking metadata. Subsequently, we will use
a code quality tool to identify technical debt items such as code smells with
all revisions surrounding each feature implementation. Lastly, we will correlate
the data sources gathered and verify whether technical debt items have an impact
on work effort feature implementation.

Through our contributions, we hope to help developers discover bottlenecks in
productivity, managers to allocate appropriate resources for refactoring
activities and business stakeholders to become more aware of development
concerns.

% \end{document}  % This is where a 'short' article might terminate


% \begin{acks}
%   The authors would like to thank Dr. Yuhua Li for providing the
%   MATLAB code of the \textit{BEPS} method.

%   The authors would also like to thank the anonymous referees for
%   their valuable comments and helpful suggestions.
% \end{acks}
