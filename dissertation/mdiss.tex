%% example.tex % Jeremy Singer % 16 Oct 12

\documentclass{mpaper}

\begin{document}

\title{Is Technical Debt Real?}
\author{Ovidiu Popoviciu}
\matricnum{2036725p}


\maketitle


\begin{abstract}
Simple abstract describing the problem, motivation, experimental design,
evaluation result and relevance.
\end{abstract}

%%%%%%%%%%%%%%%%%%%%%%%%%%%%%%%%%%%%%%%%%%%%%%%%%%%%%%%%%%%%%%%%%%%%%%%%%%%%%%%
%%%%%%%%%%%%%%%%%%%%%%%%%%%%%%%%%%%%%%%%%%%%%%%%%%%%%%%%%%%%%%%%%%%%%%%%%%%%%%%
%%%%%%%%%%%%%%%%%%%%%%%%%%%%%%%%%%%%%%%%%%%%%%%%%%%%%%%%%%%%%%%%%%%%%%%%%%%%%%%
\section{Introduction}
\label{introduction}

\begin{enumerate}
  \item General description of the problem, motivation, relevance
  \item Contributions to state of the art 
  \item Research questions
  \item Section descriptions
\end{enumerate}

% The precise structure of an MSci paper is not mandated, but it should probably
% cover in detail the following aspects of the project. \begin{enumerate} \item
% General description of the problem, motivation, relevance \item Background
% information, possibly including a literature survey \item Description of
% approach taken to solve the problem, including high-level design and
% lower-level implementation details as appropriate \item Evaluation,
% qualitative or quantitative as appropriate \item Conclusion, including scope
% for future work \end{enumerate}

%%%%%%%%%%%%%%%%%%%%%%%%%%%%%%%%%%%%%%%%%%%%%%%%%%%%%%%%%%%%%%%%%%%%%%%%%%%%%%%
%%%%%%%%%%%%%%%%%%%%%%%%%%%%%%%%%%%%%%%%%%%%%%%%%%%%%%%%%%%%%%%%%%%%%%%%%%%%%%%
%%%%%%%%%%%%%%%%%%%%%%%%%%%%%%%%%%%%%%%%%%%%%%%%%%%%%%%%%%%%%%%%%%%%%%%%%%%%%%%
\section{Background}
\label{background}

\begin{enumerate}
  \item literature review (a bit more compressed compared to proposal)
\end{enumerate}

%%%%%%%%%%%%%%%%%%%%%%%%%%%%%%%%%%%%%%%%%%%%%%%%%%%%%%%%%%%%%%%%%%%%%%%%%%%%%%%
%%%%%%%%%%%%%%%%%%%%%%%%%%%%%%%%%%%%%%%%%%%%%%%%%%%%%%%%%%%%%%%%%%%%%%%%%%%%%%%
%%%%%%%%%%%%%%%%%%%%%%%%%%%%%%%%%%%%%%%%%%%%%%%%%%%%%%%%%%%%%%%%%%%%%%%%%%%%%%%
\section{Methodology}
\label{methodology}

In this paper, we present an approach for aggregating development team data
sources such as project management tools, version control logs and static
analysis results to produce a timeline of technical debt and work effort over
the evolution of a software product. Our motivation is to empirically find a
correlation between technical debt and the amount of work effort involved in
development of work items. In the absence of work tracking information,
aggregation of such data sources might provide a relatively accurate estimation
of the number of hours a developer has put in.

When developing our approach, we completed the following steps:
\begin{enumerate}
  \item Designing an initial data model.
  \item Selecting data candidates (projects) that satisfy specific criteria.
  \item Collecting data from issue tracker, version control logs and generating
  static analysis output.
  \item Processing and refining data for analysis. 
\end{enumerate}

Section \ref{experimental-design} dives into the data model while section
\ref{data-selection} describes the data candidates criteria and selected
projects. The data collection process is described in Section
\ref{data-collection} while the calculation of work effort and technical debt is
described in Section \ref{data-processing}.

% - - - - - - - - - - - - - - - - - - - - - - - - - - - - - - - - - - - - - - -
\subsection{Experimental Design}
\label{experimental-design}

The initial step was to design a model of the aggregated data and to understand
what type of information each selected data source will provide. We consider the
following data sources relevant to the collection of data:

\begin{itemize}
  \item A \emph{Version Control System} is a system that manages changes to the
  source code. As work items are implemented, the system logs all changes made
  by the development team. The most popular version control tool is Git
  \emph{REFERENCE}. GitHub \emph{REFERENCE} and BitBucket \emph{REFERENCE} are
  hosting services for source code management.

  \item \emph{Project Management Tools} are software systems that help teams
  track, manage and collaborate on various types of units of work. A common
  example of a complex project management tool is Jira \emph{REFERENCE}. GitHub
  and BitBucket also provide an issue tracking tool with each code repository,
  although they do not provide such extensive features as Jira.
  
  \item \emph{Static Analysis Tools} are tools that analyse the source code to
  check for quality issues, security flaws and adherence to industry
  standards.There are many examples of static analysis tools: SonarQube,
  Spotbugs (formerly Findbugs), CAST, Sonargraph, etc.

\end{itemize}

All the data sources contain detailed information related to the set of changes
that the source code has suffered, the requirement that the developer is working
on and the possible quality issues she will encounter during implementation. In
many cases, these systems have been implemented by different producers and thus
are generally ``separated''. Therefore, data must be collected from each system,
in isolation. However, forms of ``light integrations'' exists for development
productivity purposes such as linking of work items to change-sets by specifying
the work item ID in the change-set description message. Such provide guidance on
how many change-sets the source code has suffered during the implementation of a
work item.

% - - - - - - - - - - - - - - - - - - - - - - - - - - - - - - - - - - - - - - -
\subsubsection*{Version Control}
\label{version-control}

The version control system keeps track and logs all changes that a source code
has suffered over the evolution of the product. Such logs are indispensable due
to their metadata which provides information on the accomplished work.

Git \emph{REFERENCE} is a popular version control tool that tracks changes of
the source code using branches and commits. Projects are stored in a
``repository'' which contains at least one development line, called a
``branch''. The main development branch is commonly named the \emph{master}
branch. Branches contain a stream of small units of work, called ``commits'',
which are the snapshots of the source code at a particular point in time.
Additionally, each commit contains relevant metadata with the following fields:

\begin{itemize}
  \item \emph{Object ID} - is the identifier of the snapshot;
  \item \emph{Author} - the name (or username) of the developer that committed the
  change;
  \item \emph{Message} - a short description of the change-set;
  \item \emph{Timestamp} - the time when the commit was created;
  \item \emph{Diff} - the set of changes that the source code has suffered, when
  compared to the previous commit. 
\end{itemize}

If the system allows integrations with a project management tool, changes can be
directly linked to the appropriate work unit. The most common method is to
specify the work item ID in the message field of the commit ID. As a result, it
is simpler to find out which change-sets belong to a particular work item. 

For a project managed using Git, all the metadata can be easily retrieved and
snapshots can be accessed at any time. This makes Git a powerful tool in
studying the history of changes of a software product.

% - - - - - - - - - - - - - - - - - - - - - - - - - - - - - - - - - - - - - - -
\subsubsection*{Project Management}
\label{project-management}

Project management tools allows developers and project managers to keep track of
the work that must be accomplished and of the issues that have been reported by
the users of the software. 

Each work item is represented in the form of a \emph{ticket}, which is a report
of the work that needs to be completed. Depending on the workflow of the team,
tickets may go through various status transitions during their lifetime which
represent checkpoints of the work involved. A ticket may contain a lot of data,
including but not limited to:

\begin{itemize}
  \item \emph{ID} - the unique identifier of the work item in the project;
  \item \emph{Type} - the type of work e.g. Feature, Bug, Enhacement, etc.;
  \item \emph{Status} - the current status of the work e.g. Created, In
  development, In testing, Closed, etc.;
  \item \emph{Summary} - a brief description of the work;
  \item \emph{Description} - a longer, more detailed description of the work
  involved;
  \item \emph{Priority} - the importance of the work as considered by the team;
  \item \emph{Assignee} - the team member responsible for completing the work;
  \item \emph{Complexity} - measurement of how complex the work item is e.g. Story points;
  \item \emph{Timestamps} - the times when ticket was created, updated, closed
  and when it is due. 
\end{itemize}



% - - - - - - - - - - - - - - - - - - - - - - - - - - - - - - - - - - - - - - -
\subsubsection*{Static Analysis}
\label{static-analysis}

The output of static analysis tools are generally weaknesses and bugs that may
lead to vulnerabilities of the product. These weaknesses are the ``lowest
level'' of technical debt and the ones which developers have to encounter
daily \emph{REFERENCE}. Therefore, the output of analysis tools is vital to
the calculation of technical debt.

Although static analysis tools provide an approximate measure of the quality of
the code base, there is a danger in associating technical debt with their
output. As mentioned in section \ref{background}, numerous types of technical
debt exist that could impact the development effort of a work item. For example,
static analysis tools cannot predict that the requirements gathering phase was
not completed appropriately.

\begin{enumerate}
  \item Describe the data sources to be aggregated e.g. issue tracker, version
  control, static analysis
  \item Describe the data model
\end{enumerate}

% - - - - - - - - - - - - - - - - - - - - - - - - - - - - - - - - - - - - - - -
% - - - - - - - - - - - - - - - - - - - - - - - - - - - - - - - - - - - - - - -
% - - - - - - - - - - - - - - - - - - - - - - - - - - - - - - - - - - - - - - -
\subsection{Candidates selection}
\label{data-selection}

This section describes the projects that were selected and the criteria that
went into the thought process when selecting such items.

% - - - - - - - - - - - - - - - - - - - - - - - - - - - - - - - - - - - - - - -
% - - - - - - - - - - - - - - - - - - - - - - - - - - - - - - - - - - - - - - -
% - - - - - - - - - - - - - - - - - - - - - - - - - - - - - - - - - - - - - - -
\subsection{Data collection}
\label{data-collection}

Describe the data model and low level implementation details of the data
collection process.

\begin{enumerate}
  \item Discuss the data model with entities such as repository, commit, issue,
  bug, etc. 
  \item Discuss the data gathering process. What fields were gathered? What is
  the difference between Github issues and Jira issues?
  \item Mention data sources, tools, optimisations for data candidates
\end{enumerate}

% intro
In this case, some of the data sources are distributed and use
different underlying technologies whereas others must be generated if non
existent. Data is queried from these remote systems and aggregated into a single
source of truth.

% body

% conclusion
Furthermore, to understand the correlation between work effort
and technical debt we process the data further, as explained in Section
\ref{data-processing}.


% - - - - - - - - - - - - - - - - - - - - - - - - - - - - - - - - - - - - - - -
\subsection{Data processing}
\label{data-processing}

Describe the data processing implementation details for calculating technical
debt, work effort using both commits and tickets.

\begin{itemize}
  \item Calculation of work effort: using commit timestamps and ticket timestamps.
  \item Calculation of technical debt: added, removed, total pain.
\end{itemize}

%%%%%%%%%%%%%%%%%%%%%%%%%%%%%%%%%%%%%%%%%%%%%%%%%%%%%%%%%%%%%%%%%%%%%%%%%%%%%%%
%%%%%%%%%%%%%%%%%%%%%%%%%%%%%%%%%%%%%%%%%%%%%%%%%%%%%%%%%%%%%%%%%%%%%%%%%%%%%%%
%%%%%%%%%%%%%%%%%%%%%%%%%%%%%%%%%%%%%%%%%%%%%%%%%%%%%%%%%%%%%%%%%%%%%%%%%%%%%%%
\section{Results}
\label{results}

Provide the results of the data processing step. Aggregate the data into tables
based on the repositories chosen and provide a couple of example graphs. 

%%%%%%%%%%%%%%%%%%%%%%%%%%%%%%%%%%%%%%%%%%%%%%%%%%%%%%%%%%%%%%%%%%%%%%%%%%%%%%%
%%%%%%%%%%%%%%%%%%%%%%%%%%%%%%%%%%%%%%%%%%%%%%%%%%%%%%%%%%%%%%%%%%%%%%%%%%%%%%%
%%%%%%%%%%%%%%%%%%%%%%%%%%%%%%%%%%%%%%%%%%%%%%%%%%%%%%%%%%%%%%%%%%%%%%%%%%%%%%%
\section{Discussion}
\label{discussion}

Discuss the results and limitations of the experiment.

% - - - - - - - - - - - - - - - - - - - - - - - - - - - - - - - - - - - - - - -
\subsection{Threats to validity}
\label{validity}

Risks to the methods of extraction, calculation and aggregation of technical
debt, work effort, ticket complexity, etc.


%%%%%%%%%%%%%%%%%%%%%%%%%%%%%%%%%%%%%%%%%%%%%%%%%%%%%%%%%%%%%%%%%%%%%%%%%%%%%%%
%%%%%%%%%%%%%%%%%%%%%%%%%%%%%%%%%%%%%%%%%%%%%%%%%%%%%%%%%%%%%%%%%%%%%%%%%%%%%%%
%%%%%%%%%%%%%%%%%%%%%%%%%%%%%%%%%%%%%%%%%%%%%%%%%%%%%%%%%%%%%%%%%%%%%%%%%%%%%%%
\section{Future Work}
\label{future-work}

Describe the items that could be improved and expand on the next steps within
this area of research.

%%%%%%%%%%%%%%%%%%%%%%%%%%%%%%%%%%%%%%%%%%%%%%%%%%%%%%%%%%%%%%%%%%%%%%%%%%%%%%%
%%%%%%%%%%%%%%%%%%%%%%%%%%%%%%%%%%%%%%%%%%%%%%%%%%%%%%%%%%%%%%%%%%%%%%%%%%%%%%%
%%%%%%%%%%%%%%%%%%%%%%%%%%%%%%%%%%%%%%%%%%%%%%%%%%%%%%%%%%%%%%%%%%%%%%%%%%%%%%%
\section{Conclusion}
\label{conclusion}

%%%%%%%%%%%%%%%%%%%%%%%%%%%%%%%%%%%%%%%%%%%%%%%%%%%%%%%%%%%%%%%%%%%%%%%%%%%%%%%
%%%%%%%%%%%%%%%%%%%%%%%%%%%%%%%%%%%%%%%%%%%%%%%%%%%%%%%%%%%%%%%%%%%%%%%%%%%%%%%
%%%%%%%%%%%%%%%%%%%%%%%%%%%%%%%%%%%%%%%%%%%%%%%%%%%%%%%%%%%%%%%%%%%%%%%%%%%%%%%
{\bf Acknowledgments.} This is optional; it is a location for you to thank
people

\bibliographystyle{abbrv}
\bibliography{mdiss}

\end{document}