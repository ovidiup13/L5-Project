%% example.tex % Jeremy Singer % 16 Oct 12

\documentclass{mpaper}

\begin{document}

\title{Is Technical Debt Real?}
\author{Ovidiu Popoviciu}
\matricnum{2036725p}


\maketitle


\begin{abstract}
Simple abstract describing the problem, motivation, experimental design,
evaluation result and relevance.
\end{abstract}

%%%%%%%%%%%%%%%%%%%%%%%%%%%%%%%%%%%%%%%%%%%%%%%%%%%%%%%%%%%%%%%%%%%%%%%%%%%%%%%
%%%%%%%%%%%%%%%%%%%%%%%%%%%%%%%%%%%%%%%%%%%%%%%%%%%%%%%%%%%%%%%%%%%%%%%%%%%%%%%
%%%%%%%%%%%%%%%%%%%%%%%%%%%%%%%%%%%%%%%%%%%%%%%%%%%%%%%%%%%%%%%%%%%%%%%%%%%%%%%
\section{Introduction}
\label{introduction}

\begin{enumerate}
  \item General description of the problem, motivation, relevance
  \item Contributions to state of the art 
  \item Research questions
  \item Section descriptions
\end{enumerate}

% The precise structure of an MSci paper is not mandated, but it should probably
% cover in detail the following aspects of the project. \begin{enumerate} \item
% General description of the problem, motivation, relevance \item Background
% information, possibly including a literature survey \item Description of
% approach taken to solve the problem, including high-level design and
% lower-level implementation details as appropriate \item Evaluation,
% qualitative or quantitative as appropriate \item Conclusion, including scope
% for future work \end{enumerate}

%%%%%%%%%%%%%%%%%%%%%%%%%%%%%%%%%%%%%%%%%%%%%%%%%%%%%%%%%%%%%%%%%%%%%%%%%%%%%%%
%%%%%%%%%%%%%%%%%%%%%%%%%%%%%%%%%%%%%%%%%%%%%%%%%%%%%%%%%%%%%%%%%%%%%%%%%%%%%%%
%%%%%%%%%%%%%%%%%%%%%%%%%%%%%%%%%%%%%%%%%%%%%%%%%%%%%%%%%%%%%%%%%%%%%%%%%%%%%%%
\section{Background}
\label{background}

\begin{enumerate}
  \item literature review (a bit more compressed compared to proposal)
\end{enumerate}

%%%%%%%%%%%%%%%%%%%%%%%%%%%%%%%%%%%%%%%%%%%%%%%%%%%%%%%%%%%%%%%%%%%%%%%%%%%%%%%
%%%%%%%%%%%%%%%%%%%%%%%%%%%%%%%%%%%%%%%%%%%%%%%%%%%%%%%%%%%%%%%%%%%%%%%%%%%%%%%
%%%%%%%%%%%%%%%%%%%%%%%%%%%%%%%%%%%%%%%%%%%%%%%%%%%%%%%%%%%%%%%%%%%%%%%%%%%%%%%
\section{Methodology}
\label{methodology}

In this paper, we present an approach for aggregating development team data
sources such as project management tools, version control logs and static
analysis results to produce a timeline of technical debt and work effort over
the evolution of a software product. Our motivation is to empirically find a
correlation between technical debt and the amount of work effort involved in
development of work items. In the absence of work tracking information,
aggregation of such data sources might provide a relatively accurate estimation
of the number of hours a developer has put in.

When developing our approach, we completed the following steps:
\begin{enumerate}
  \item Designing an initial data model.
  \item Selecting data candidates (projects) that satisfy specific criteria.
  \item Collecting data from issue tracker, version control logs and generating
  static analysis output.
  \item Processing and refining data for analysis. 
\end{enumerate}

Section \ref{experimental-design} dives into the data model while section
\ref{data-selection} describes the data candidates criteria and selected
projects. The data collection process is described in Section
\ref{data-collection} while the calculation of work effort and technical debt is
described in Section \ref{data-processing}.

% - - - - - - - - - - - - - - - - - - - - - - - - - - - - - - - - - - - - - - -
\subsection{Experimental Design}
\label{experimental-design}

The initial step was to design a model of the aggregated data and to understand
what information is required from each data source. We considered the following
data sources relevant to the collection of data:

\begin{itemize}
  \item A \emph{Version Control System} is a system that manages changes to the
  source code. As work items are implemented, the system logs all changes made
  by the development team. Such logs are indispensable because they provide
  detailed information on the work item associated with the change, author,
  timestamps and number of lines added, deleted or modified. The most popular
  version control tool is Git \emph{REFERENCE}. GitHub \emph{REFERENCE} and
  BitBucket \emph{REFERENCE} are hosting services for source code management.

  \item \emph{Project Management Tools} are software systems that help teams
  track, manage and collaborate on various types of units of work. A report of
  the work item is represented in the form of a \emph{ticket} and contains
  information on its status, assignee, complexity, deadlines, comments and many
  more. The metadata of the ticket may contain other information such as status
  transition timestamps, which gives an estimate of the time when a work item
  has started and finished. A common example of a complex project management
  tool is Jira \emph{REFERENCE}. GitHub and BitBucket also provide an issue
  tracking tool with each code repository, although they do not provide such
  extensive features as Jira.
  
  \item \emph{Static Analysis Tools} are tools that analyse the source code to
  check for quality issues, security flaws and adherence to industry standards.
  The output of static analysis tools are generally weaknesses and bugs that may
  lead to vulnerabilities of the product. These weaknesses are the ``lowest
  level'' of technical debt and the ones which developers have to encounter
  daily. Therefore, the output of analysis tools is vital to the calculation of
  technical debt. There are many examples of static analysis tools: SonarQube,
  Spotbugs (formerly Findbugs), CAST, Sonargraph, etc.

\end{itemize}

Such changes can be directly linked to the appropriate work item in the project
management tool.

Although static analysis tools provide an approximate measure of the quality of
the code base, there is a danger in associating technical debt with their
output. As mentioned in section \ref{background}, numerous types of technical
debt exist that could impact the development effort of a work item. For example,
static analysis tools cannot predict that the requirements gathering phase was
not completed appropriately. 

\begin{enumerate}
  \item Describe the data sources to be aggregated e.g. issue tracker, version
  control, static analysis
  \item Describe the data model
\end{enumerate}

% - - - - - - - - - - - - - - - - - - - - - - - - - - - - - - - - - - - - - - -
\subsection{Candidates selection}
\label{data-selection}

This section describes the projects that were selected and the criteria that
went into the thought process when selecting such items.

% - - - - - - - - - - - - - - - - - - - - - - - - - - - - - - - - - - - - - - -
\subsection{Data collection}
\label{data-collection}

Describe the data model and low level implementation details of the data
collection process.

\begin{enumerate}
  \item Discuss the data model with entities such as repository, commit, issue,
  bug, etc. 
  \item Discuss the data gathering process. What fields were gathered? What is
  the difference between Github issues and Jira issues?
  \item Mention data sources, tools, optimisations for data candidates
\end{enumerate}

% intro
In this case, some of the data sources are distributed and use
different underlying technologies whereas others must be generated if non
existent. Data is queried from these remote systems and aggregated into a single
source of truth.

% body

% conclusion
Furthermore, to understand the correlation between work effort
and technical debt we process the data further, as explained in Section
\ref{data-processing}.


% - - - - - - - - - - - - - - - - - - - - - - - - - - - - - - - - - - - - - - -
\subsection{Data processing}
\label{data-processing}

Describe the data processing implementation details for calculating technical
debt, work effort using both commits and tickets.

\begin{itemize}
  \item Calculation of work effort: using commit timestamps and ticket timestamps.
  \item Calculation of technical debt: added, removed, total pain.
\end{itemize}

%%%%%%%%%%%%%%%%%%%%%%%%%%%%%%%%%%%%%%%%%%%%%%%%%%%%%%%%%%%%%%%%%%%%%%%%%%%%%%%
%%%%%%%%%%%%%%%%%%%%%%%%%%%%%%%%%%%%%%%%%%%%%%%%%%%%%%%%%%%%%%%%%%%%%%%%%%%%%%%
%%%%%%%%%%%%%%%%%%%%%%%%%%%%%%%%%%%%%%%%%%%%%%%%%%%%%%%%%%%%%%%%%%%%%%%%%%%%%%%
\section{Results}
\label{results}

Provide the results of the data processing step. Aggregate the data into tables
based on the repositories chosen and provide a couple of example graphs. 

%%%%%%%%%%%%%%%%%%%%%%%%%%%%%%%%%%%%%%%%%%%%%%%%%%%%%%%%%%%%%%%%%%%%%%%%%%%%%%%
%%%%%%%%%%%%%%%%%%%%%%%%%%%%%%%%%%%%%%%%%%%%%%%%%%%%%%%%%%%%%%%%%%%%%%%%%%%%%%%
%%%%%%%%%%%%%%%%%%%%%%%%%%%%%%%%%%%%%%%%%%%%%%%%%%%%%%%%%%%%%%%%%%%%%%%%%%%%%%%
\section{Discussion}
\label{discussion}

Discuss the results and limitations of the experiment.

% - - - - - - - - - - - - - - - - - - - - - - - - - - - - - - - - - - - - - - -
\subsection{Threats to validity}
\label{validity}

Risks to the methods of extraction, calculation and aggregation of technical
debt, work effort, ticket complexity, etc.


%%%%%%%%%%%%%%%%%%%%%%%%%%%%%%%%%%%%%%%%%%%%%%%%%%%%%%%%%%%%%%%%%%%%%%%%%%%%%%%
%%%%%%%%%%%%%%%%%%%%%%%%%%%%%%%%%%%%%%%%%%%%%%%%%%%%%%%%%%%%%%%%%%%%%%%%%%%%%%%
%%%%%%%%%%%%%%%%%%%%%%%%%%%%%%%%%%%%%%%%%%%%%%%%%%%%%%%%%%%%%%%%%%%%%%%%%%%%%%%
\section{Future Work}
\label{future-work}

Describe the items that could be improved and expand on the next steps within
this area of research.

%%%%%%%%%%%%%%%%%%%%%%%%%%%%%%%%%%%%%%%%%%%%%%%%%%%%%%%%%%%%%%%%%%%%%%%%%%%%%%%
%%%%%%%%%%%%%%%%%%%%%%%%%%%%%%%%%%%%%%%%%%%%%%%%%%%%%%%%%%%%%%%%%%%%%%%%%%%%%%%
%%%%%%%%%%%%%%%%%%%%%%%%%%%%%%%%%%%%%%%%%%%%%%%%%%%%%%%%%%%%%%%%%%%%%%%%%%%%%%%
\section{Conclusion}
\label{conclusion}

%%%%%%%%%%%%%%%%%%%%%%%%%%%%%%%%%%%%%%%%%%%%%%%%%%%%%%%%%%%%%%%%%%%%%%%%%%%%%%%
%%%%%%%%%%%%%%%%%%%%%%%%%%%%%%%%%%%%%%%%%%%%%%%%%%%%%%%%%%%%%%%%%%%%%%%%%%%%%%%
%%%%%%%%%%%%%%%%%%%%%%%%%%%%%%%%%%%%%%%%%%%%%%%%%%%%%%%%%%%%%%%%%%%%%%%%%%%%%%%
{\bf Acknowledgments.} This is optional; it is a location for you to thank
people

\bibliographystyle{abbrv}
\bibliography{mdiss}

\end{document}