\documentclass{mprop}
\usepackage{graphicx}

% alternative font if you prefer
%\usepackage{times}

% for alternative page numbering use the following package
% and see documentation for commands
%\usepackage{fancyheadings}


% other potentially useful packages
%\uspackage{amssymb,amsmath}
\usepackage{url}
%\usepackage{fancyvrb}
%\usepackage[final]{pdfpages}

\begin{document}

%%%%%%%%%%%%%%%%%%%%%%%%%%%%%%%%%%%%%%%%%%%%%%%%%%%%%%%%%%%%%%%%%%%
\title{Technical Debt Title Here}
\author{Ovidiu Popoviciu}
\date{18th December 2017}
\maketitle
%%%%%%%%%%%%%%%%%%%%%%%%%%%%%%%%%%%%%%%%%%%%%%%%%%%%%%%%%%%%%%%%%%%

%%%%%%%%%%%%%%%%%%%%%%%%%%%%%%%%%%%%%%%%%%%%%%%%%%%%%%%%%%%%%%%%%%%
\tableofcontents
\newpage
%%%%%%%%%%%%%%%%%%%%%%%%%%%%%%%%%%%%%%%%%%%%%%%%%%%%%%%%%%%%%%%%%%%

%%%%%%%%%%%%%%%%%%%%%%%%%%%%%%%%%%%%%%%%%%%%%%%%%%%%%%%%%%%%%%%%%%%
\section{Introduction}\label{intro}

\begin{itemize}
	\item What is technical debt?
	\item What are its applications?
	\item Why is it important?
\end{itemize}

Technical debt - is it real? How does it correlate with goals of the sprint?\\

Problem statement: Has the student analysed the problem, stated it clearly, and justified its importance?

%%%%%%%%%%%%%%%%%%%%%%%%%%%%%%%%%%%%%%%%%%%%%%%%%%%%%%%%%%%%%%%%%%%
\section{Statement of Problem}

Clearly state the problem to be addressed in your forthcoming project.\\

Explain why it would be worthwhile to solve this problem. \\

%%%%%%%%%%%%%%%%%%%%%%%%%%%%%%%%%%%%%%%%%%%%%%%%%%%%%%%%%%%%%%%%%%%
\section{Background Survey}
\subsection{Definition}

% Ward Cunningham - WyCash Portfolio Management System
Technical debt is a metaphor termed by Ward Cunningham, in his famous report on the WyCash Portfolio Management System in 1993 \cite{Cunningham1993}.
In the report, Cunningham mentioned that "\textit{shipping first time code is like going into debt.}" and that as the system evolves new features would become more and more difficult to implement.
This phenomenon was due to feature-rich projects being shipped to customers early but poorly written with little or no consideration to quality and to future work.

\begin{figure}
	\centering
	\includegraphics[width=\linewidth]{visualisations/TD_trend.png}
	\caption{Technical Debt Trend from 2004 to Present}
	\label{fig:td-trend}
\end{figure}

% MTD 2010
The metaphor was ignored for a long time, until the late 2000s, when more and more studies started to look into the metaphor and explore its definition, identification process, measurement and management.
A figure of the popularity of Technical Debt from 2004 to present can be seen in Figure \ref{fig:td-trend}, information provided by Google Trends \cite{GoogleTrends}.
Thus, the first workshop on managing technical debt took place in 2010, where an initial research agenda was proposed for the future of software engineering field.
Since then, workshops have been held every year, which consisted seminars, presentations and brainstorming sessions on aspects such as
definition \cite{Kruchten2012} \cite{Theodoropoulos2011} \cite{Schmid2013},
identification \cite{Ernst2012},
measurement \cite{Letouzey2012} \cite{Curtis2012} \cite{Nugroho2011} \cite{Zazworka2011} \cite{Fontana2012} \cite{Bohnet2011},
management \cite{Guo2011} \cite{Zazworka2011Prioritise} \cite{Seaman2012} to
industry case studies \cite{Lim2012} \cite{Morgenthaler2012} \cite{Codabux2013} \cite{Holvitie2014} \cite{Klinger2011}.

% from metaphor to theory and practice
% TODO: define principal and interest
The definition of technical debt relies heavily on the perspective of the viewer and her responsibility within the project domain.
This perspective encompasses the entire socio-technical environment, starting with developers who deal with it on a daily basis, to architects, project managers and stakeholders.
Kurchten et al. \cite{Kruchten2012} defined technical debt as technological gaps between development teams, product managers and stakeholders, where a gap is an evolution of a context specific to a particular decision taken in the past.
These gaps might have been decisions that seemed correct but with the passing of time, the initial decision had incurred debt within the project.
As a consequence, the authors stated that technical debt was not a mere association between the results of static code analyzers but heavily reliant on project evolution and future decision management.


% from a stakeholder's perspective
% TODO: talk about non-existent vocabulary
However, future decisions of project are influenced by stakeholders of the business.
Unfortunately, management might or might not have knowledge of such an important aspect of their project.
Although it is directly related to quality, stakeholders generally care little about it unless it is, as put by Theodoropoulos et al. \cite{Theodoropoulos2011}, an extrinsic (or visible) quality of software with a direct impact on business revenue.
For example, an extrinsic quality characteristic is usability. Deferring user experience issues might force users of the software to find "ways around" certain tasks which negatively impact user productivity and usefulness of the software provided.
Such characteristics are important to the business as they are "sell points" of the product.
On the other hand, intrinsic characteristics of software are considered "low level", code based issues such as code smells, best-practices violations that slow down development over time if not managed adequately.
Additionally, Theodoropoulos et al. \cite{Theodoropoulos2011} have considered that these software quality characteristics are interdependent, and thus deferring quality maitainence in one area may affect other areas of quality.
An interesting idea of the authors was that technical debt should only be associated with quality maintainance costs only within technology environment rather than business and process environment.
Thus the term \textbf{technical} debt.

% on the limits of td metaphor
Although the use of finance terms may simplify technical characteristics of software quality in the dialogue between development teams and stakeholders, the analogy breaks down as studied by Schimd et al. \cite{Schmid2013}.
In their study, the authors had identified shortcomings in the financial metaphor established by Ward Cunningham \cite{Cunningham1993} and found points where it breaks down.
In the financial domain, debt is a well known arrangement between two parties where one party borrowes a fixed amount of money from the other party \cite{debt-investopedia}.
The most well known types of debt are loans, where the terms of the arrangement dictate that the amount of money borrowed must be paid back in full after a fixed period of time, along with fixed interest payments paid annually.

Schmid et al. \cite{Schmid2013} has identified three major points where the analogy breaks down:
\begin{itemize}
	\item \textit{Unit of measurement}. In finance there is a clear unit of monetary measurement through the use of international currencies.
	In contrast, technical debt does have a standard unit of measurement defined. 
	There are many tools (TODO: add citation of paper tools here) that provide a single, quantified and aggregated measure of the amount of time it takes for software quality issues to be resolved.
	Additionally, very few of these tools quantify the consequences of neglecting refactoring and the improvement in quality characteristics.
	The issue of measurement puts a dent into the shared vocabulary between development and stakeholders as mentioned by a number of software practioners from industry case studies (TODO: add industry case study citation here).
	\item \textit{Fixed time period}. On the one hand, debt arrangements have a fixed \textit{maturity date}, where the debt must be paid back.
	On the other hand, software quality issues do not a have a time limit and may be kept in the product until they are resolved.
	Fixing these items relate to paying back the "loan" taken on their creation.
	Such issues may never be paid back if not needed to, resulting in an increased cost-value ratio.
	\item \textit{Fixed interest}. A loan arrangement additionally consists of fixed interest payments measured as a percetange of the loan value, paid on an annual or bi-annual basis.
	The interest compensates for the risk taken by the lender and encourages the loanee to pay back quickly as possible in order to avoid paying back too much interest.
	In reality, there is no such thing as fixed interest dependant on a single factor in technical debt.
	Interest is difficult to quantify as a matter of principal, and the amount of interest paid after a period of time depends on future work.
\end{itemize}
Under these circumstances, what is considered "good structure" or "clean code" is also heavily influenced to future development since future decisions influence cost impact.
As a consequence, no system is \textit{debt free} and thus fixing every code violation would be considered gold-plating.


% martin fowler - td quadrants
In Martin Fowler's famous article \cite{TDMartin}, he described this type of future debt as inadeverted-prudent debt.
Over time, a project that was "clean" may find out after a period of time that the initial approach taken might not have been the best.
He considered that developers learn on the job to perfect their craft as time passes.
The four quadrants refer to the types of technical debt that one might encounter in a software project given the approach taken by the development team.
It was one of the four quandrants he defined, as shown in Figure \ref{fig:td-quandrants}.

\begin{figure}
	\centering
	\includegraphics[width=0.5\linewidth]{visualisations/TD_quadrants.png}
	\caption{Martin Fowler's Technical Debt Quadrants}
	\label{fig:td-quandrants}
\end{figure}


% unhedged call option
An alternative metaphor of technical debt \cite{UnhedgedCallOption}, described bad code using finance terms in a similar fashion, but through a different financial intrument called a call option.
\textit{An option is a financial derivative that represents a contract sold by one party (the option writer) to another party (the option holder). The contract offers the buyer the right, but not the obligation, to buy (call) or sell (put) a security or other financial asset at an agreed-upon price (the strike price) during a certain period of time or on a specific date (exercise date).} \cite{option-investopedia}.
A call option gives the right to buy while a put option gives the option holder the right to sell.
In software engineering, if a feature is hacked up quickly using bad code and never touched again, then the project had reaped the rewards.
The option "was not called".
However, if a new feature were required that would be influenced by the quick and dirty work implemented earlier, then the requirement would be more expensive to fulfill.
In this case, the option "has been called".



%%%%%%%%%%%%%%%%%%%%%%%%%%%%%%%%%%%%%%%%%%%%%%%%%%%%%%%%%%%%%%%%%%%
\subsection{Identification}

\subsubsection{On the role of Requirements in Understanding and Managing Technical Debt} \cite{Ernst2012}
\begin{itemize}
	\item What were the goals of the paper? \\
	      The goal of the paper was to describe technical debt within project requirements, including gathering and change management phases.
	\item What was the methodology? \\
	      The authors created a tool, RE-KOMBINE, that structure requirements as goals, targets and desired properties of the system.
	      They are bucketed into high-level and low-level objectives, and each had an assigned set of tasks which led to a solution.
	      The system had the ability to manage evolving requirements by changing the assigned tasks of the goals.
	\item What did the authors learn?\\
	      The relationship between TD in requirements and TD in implementation can be described in terms of product value.
	      If the product does not bring value to customers, implementation TD does not matter.
	      If implementation TD is high, then delivering value is extremely difficult.\\
	      The authors have defined TD in requirements as follows:\\
	      \textit{Technical debt in requirements process is the distance between the original solution S for a requirement R, and the changed requirement R} \\
	      The conclusions of the paper was that requirements must be refined at the very start of the project/work week/sprint so that no extra development effort is put into satisfying it.
	      If the requirements are not gathered correctly then implementation may not bring value to the customers.
	      The proposed solution was to insist on requirements forecasting from an early stage at the expense of extra costs.
	\item What were the limitations of the study?\\
	      The system has not been evaluated in the industry environment.\\
	      Requirements TD is not TD per se, but a measure of how the system changes over time with regards to its requirements gathering process.
	\item Links to other resources? \\
\end{itemize}


%%%%%%%%%%%%%%%%%%%%%%%%%%%%%%%%%%%%%%%%%%%%%%%%%%%%%%%%%%%%%%%%%%%
\subsection{Measurement}

\subsubsection{The SQALE Method for Measuring Technical Debt} \cite{Letouzey2012}

\begin{itemize}
	\item What were the goals of the paper? \\
	      The goal of this paper was to propose a standardized, language-agnostic framework for assessing the quality of source code by deriving measures for important code characteristics and calculating technical debt.
	\item What was the methodology? \\
	      The framework proposed consisted of four concepts:
	      \begin{itemize}
		      \item Quality Model - defines internal properties of code through a structured three-layer hierarchy (characteristic, sub-characteristic and requirement).
		      \item Analysis Model - measurement of the distance between the current state of the application and the \"optimized\" state, the quality target.
		      \item Indices - each characteristic of the Quality Model defines a remediation index (cost to repair the non-compliances) defined in time, work or capital units.
		      \item Indicators - provide a visual representation of technical debt either through ratings (ratios between TD and development cost) or SQALE Pyramids (distribution of TD over all the characteristics).
	      \end{itemize}
	\item What did the authors learn?
	      \begin{itemize}
		      \item Each organization must manage technical debt as early as possible in a project with indicators and dashboards for code quality available for each build within the continous integration pipeline.
		      \item Quality model must be calibrated according to the requirements and policies of the organization implementing the framework. For example, to approve/decline rules that are considered debt and what principal costs are for each type of code violation.
		      \item One limitation is that the framework does not take into consideration other types of technical debt such as requirements debt, operational debt.
		      \item There is no standardized definition of right code, organizations have to define their own \"right\" code rules.
	      \end{itemize}
	\item Links to other resources?
	      -
\end{itemize}

\subsubsection{Estimating the Size, Cost and Types of Technical Debt} \cite{Curtis2012}
\begin{itemize}
	\item What were the goals of the paper? \\
	      Summarized the results of a study on a huge database (365M LOC) of software projects across 10 industries.
	      The study was language agnostic and provided a formula for estimating the principal of TD items.
	\item What was the methodology? \\
	      Made use of CAST's Application Intelligence platform to statically analyse source code, using 1200 rules for good practices.
	      The steps taken were as follows: parsing, identify violations, aggregate results and sum up into a quality characteristics, or \textit{health factors}.
	      Scores from each health factor were aggregated on a scale of 1 (high risk) to 4 (low risk).
	      The principal was estimated by a formula with three parameters: number of problems, time required for each fix and the cost of fixing the issue.
	\item What were the conclusions? \\
	      The authors concluded that estimating the interest incurred by a TD item was difficult since might be multiple hidden factors which influenced the results.
	      The term works well with the phenomenon since stakeholders think of software quality in terms of business. \\
	      There are trade-offs of TD management from stakeholder's perspective.
	      For example, fixing TD items within the Robustness characteristic may lead to fewer operational failures and higher availability of the system.
	\item What were the limitations? \\
	\item Links to other resources? \\
\end{itemize}

\subsubsection{An Empirical Model of Technical Debt and Interest}  \cite{Nugroho2011}

\begin{itemize}
	\item What were the goals of the paper? \\
	      The goal was to formally define technical debt and interest to provide insights on the Return On Investment (ROI) of IT executives.
	      Specifically, it tried to answer the following questions from a financial perspective: How large is TD? How much interest are paying? What are the consequences of holding on to debt?
	\item What was the methodology? \\
	      The authors had completed an empirical analysis of 44 projects within the Software Improvement Group (SIG) using TUViT software quality assessment method for collecting relevant metrics: lines of code, code duplications, etc.
	      The had also defined technical debt as being the changes needed to bring a system from its current quality state to the "ideal" quality. This can be quantified as the repair effort (RE) and can be calculated as follows:
	      RE = Rework Fraction (LOC need to be changed) * Rebuild Value (estimate in man-months of rebuilding with a particular technology) * Refactoring Adjustment (advanced tooling helping the team be more productive.)\\
	      The interest was also derived from the extra cost spent on maintainance per month and per year, modelled against the quality factor (the current state of quality within the system).
	\item What did the authors learn?
	      Maintainance was considered a different activity by the authors since the tasks involved in maintaining a system involve a change that is immediately visible from the outside.
	      However, technical debt repair was not considered a maintainance activity.
	      A few limitations of this approach were the level of precision rating of a system (quality ratings are on a scale of 1-5) which leads to less accurate estimations on important financial numbers such as ROI.
	      Refactoring adjustment variable is subjective as it requires expert input when considering development productivity.
	\item Links to other resources? \\
	      - S. McConnell. 10x software development \\
	      - Maintainability measurement - I. Heitlager, T. Kuipers, and J. Visser. A practical model for measuring maintainability. In Quality of Information and Communications Technology, 2007. QUATIC 2007. 6th International Conference on the, pages 30–39. IEEE, 2007.
	      - Eick, T. Graves, A. Karr, J. Marron, and A. Mockus. Does code decay? assessing the evidence from change management data. Software Engineering, IEEE Transactions on, 27(1):1–12, 2002.
\end{itemize}

\subsubsection{Investigating the Impact of Design Debt on Software Quality} \cite{Zazworka2011}
\begin{itemize}
	\item What were the goals of the paper? \\
	      To investigate the impact God classes have on development and whether they are points of major changes and defects when compared to other parts of the system. Additionally, to study whether there is a linear relationship between class size, change and defect likelihoods and how the results are influenced by data normalizations

	\item What was the methodology?
	      The authors ran an analysis on two medium-sized projects from a small development company. The analysis consisted of statically analysing the source code stored in version control and the project issues from the project management system.
	      It looked at two major variables:
	      \begin{itemize}
		      \item Change likelihood = how likely code within and outside God classes changes accross revisions.
		      \item Defect likelihood = how likely is a fix implemented inside and outside of God classes.
	      \end{itemize}
	\item What did the authors learn?
	      Empirical analysis of the two projects has shown that God classes get changed 7.8\% of the time whilst also being fixed 17 times more than the rest of the code.
	      However, since the God classes have more functionality included, normalization of the data by LOC has shown that there are no significant results in comparing God vs non-God classes.
	      The limited scope of the empirical analysis on a sample of two projects and focus on a specific code smell, the results are indecisive for a generalization of the results.
	\item Links to other resources?
\end{itemize}

\subsubsection{Investigating the impact of Code Smells Debt on Quality Code Evaluation} \cite{Fontana2012}
\begin{itemize}
	\item What were the goals of the paper? \\
	      The goals were to study the impact on code quality metrics of removing code smell, which code smell incurs the most TD and whether their impact on is related to the domain of the software application.
	      Should code smells be categorized as design debt?
	\item What was the methodology? \\
	      The study looked at three most common smells: Data class, God class and Duplicated Code.
	      The metrics impacted are related to cohesion, coupling and complexity; calculated by various tools according to clarity of computation.
	      Best refactoring practices were applied for each smell detected in the system and the quality metrics have been re-analysed to assess their impact.
	\item What did the authors learn? \\
	      Some code quality tools evaluate code smells through a set of rules which must be customized according to the application domain of the system.
	      Refactoring of one code smell may provide benefits for one or more metric qualities but may negatively impact others.
	      Additionally, code duplication is one of the worst smells but may also provide benefits in some cases.
	      Data Class smell and God class smells may be domain-dependent, but this assumption heavily correlates with the types of tools and libraries listed as dependencies in the system.
	\item Links to other resources? \\
	      Zhang et al. investigated relationship between code smells and software faults => how to prioritize refactoring.\\
	      Arcelli et al. proposed a first analysis on refactoring on a small scale of software metrics.
\end{itemize}

\subsubsection{Monitoring Code Quality and Development Activity by Software Maps} \cite{Bohnet2011}
\begin{itemize}
	\item What were the goals of the paper? \\
	      The goal of the paper was to bridge the gap between development teams and corporate managers, by exposing internal system quality through the use of software maps.
	      Software maps enable managers to assess where improvement in quality is most necessary and can observe future quality risk and estimate future maintenance costs.
	      Additionally, it helps developers and project managers identify modules suitable for refactoring/clean-up before the start of a sprint.
	\item What was the methodology? \\
	      A software map is a hierarchical 2D/3D view of software artefacts within a project.
	      Each artefact is represented visually through properties such as colour, texture and size; each representing a property of quality: cyclomatic complexity, LOC, nesting levels.
	      The authors have implemented a software for visualizing software maps and evaluated the tool on two projects: JBoss and Blender.
	\item What were the conclusions? \\
	      The authors have defined the internal quality of the system to be directly correlated to the quality of the source code.
	      These qualities are invisible to customers and management, and stakeholders are reluctant to put more resources into internal quality.
	      On the other hand, external quality relates to the visible, discoverable qualities of the system.
	      For example, defects and bugs are two discoverable phenomenons of external quality.
	      Impact of external quality issues are immediate, and stakeholders are inclined to invest in quick features/fixes that make an instant impact.
	\item What were the limitations? \\
	      The system has not been evaluated in practice.\\
	      There is no aggregation of values into a single metric that is understandable by both parties.\\
	      Calculating the cost of a change is difficult, only shows the past and current quality status.
	\item Links to other resources? \\
\end{itemize}

%%%%%%%%%%%%%%%%%%%%%%%%%%%%%%%%%%%%%%%%%%%%%%%%%%%%%%%%%%%%%%%%%%%
\subsection{Management}

\subsubsection{A Portfolio Approach to Managing Technical Debt} \cite{Guo2011}

\begin{itemize}
	\item What were the goals of the paper?\\
	      To define technical debt from a financial portfolio perspective by encouraging to be viewed as an investment and therefore used as a strategy.
	      Which TD items should be paid in a release for a particular component \textit{S}?
	\item What was the methodology?\\
	      The authors were looking at debt items in a system similar to assets in a financial portfolio, looking to maximize return on investment and minimize the risk associated with TD items.
	      In short, they were interested to find out whether an optimal combination of assets (TD items) can be found. \\
	      An approach proposed was to identify historical metrics of TD items and decide which ones to keep (the ones on which to pay interest) and which ones to sell (which to pay the principal i.e. FIX).
	\item What did the authors learn? \\
	      Technical debt is not problematic unless the costs outweight the benefits.
	      The approach does not take into consideration the fact that TD items may be recklessly added to the "portfolio" and the management of undetected TD items.
	      Additionally, technical debt is not the same as financial debt. The amount of principal and interest are known before hand and fixed in finance whereas in software development estimation is one of the most difficult tasks.
	      Approach was not tested in a real setting.
	\item Links to other resources?
\end{itemize}

\subsubsection{Prioritising Design Debt Investment Opportunities} \cite{Zazworka2011Prioritise}
\begin{itemize}
	\item What were the goals of the paper? \\
	      Proposes an approach for making refactoring decision regarding God classes in software projects, that will have a low principal cost and long-term benefits.
	\item What was the methodology? \\
	      The authors have considered the cost-benefit analysis criteria for prioritization, by taking into consideration the cost of refactoring and quality gained from each refactoring of TD items.
	      The proposed method approximates size of refactoring based on source code complexity metrics such as Weighted Method Count, Tight Class Cohesion and Access to Foreign Data Metric.
	      A feasibility study was conducted on a small size software development company on a project with 35k LOC, 11 month old and mantained by a team of 4 developers.
	\item What were the limitations of the study? \\
	      There were no results related to the feasibility study.
	      The study was only conducted on a single class smell, the God class.
	      Choice of refactoring must also be evaluated by the business side.
	\item Links to other resources? \\
\end{itemize}


\subsubsection{Using Technical Debt Data in Decision Making} \cite{Seaman2012}
\begin{itemize}
	\item What were the goals of the paper? \\
	      To discuss the approaches of technical debt management and to propose four novel ways to aid decision making.
	\item What were the conclusions? \\
	      Proposed four approaches for decision making on TD:
	      \begin{itemize}
		      \item Cost-Benefit Analysis. Defined principal, interest probability (the probability that other work will be more expensive) and the interest amount.
		            Based on these three factors and their associated value (high, medium, low), the authors claimed it is possible to estimate items that have low principal (are quick to repay) and a possible high impact on future additions and changes.
		      \item Analytic Hierarchy Process (AHP). Defined a criteria by which TD items would be paid off, through group decisions.
		      \item Portfolio Approach. Guo et al. \cite{Guo2011} has defined an approach for portfolio debt management. The same approach has been defined here.
		      \item Options. Represents investments into refactoring as a purchase of options that will facilitate change in the future but with no immediate profit.
	      \end{itemize}
\end{itemize}

%%%%%%%%%%%%%%%%%%%%%%%%%%%%%%%%%%%%%%%%%%%%%%%%%%%%%%%%%%%%%%%%%%%
\subsection{Industry Case Studies}

\subsubsection{What Software Practitioners have to say about TD} \cite{Lim2012}
\begin{itemize}
	\item What were the goals of the paper? \\
	      The goals of the study was to gather information on how practitioners identify, visualize and manage technical debt in industry.
	\item What was the methodology? \\
	      The authors have had interviews with 35 practitioners with broad backgrounds: various companies, a range of years of experiences, positions, etc.
	      The interview questions were both open-ended and closed-ended.
	\item What were the conclusions? \\
	      After being introduced to a definition of TD, 25\% of the participants considered TD as unintentional, and the rest considered that TD was caused by various trade-offs.
	      They understood that business reality forces trade-offs within software quality in order to satisfy business goals.
	      Furthermore, some practitioners mentioned that their team cut back on software quality enforcement methods due to time constraints and due to this TD was incurred in their projects.
	      They had to balance requirements and software quality against deadlines, since customer/business demands always took precedence. \\

	      However, participants have struggled to find a way to measure TD and its cumulative effort over time.
	      Moreover, convincing management to invest resources in paying back TD is difficult, unless there is an associated business value.
	      Unfortunately, there was no vocabulary defined between development teams and business, hence difficult to explain TD and its associated costs over time.
	      Some teams have had success in convincing stakeholders of the impact of TD, by quickly implementing a set of requirements and visualizing their trade-offs over a period of time.
	      As a result, they reported that stakeholders were happy to negotiate extending deadlines for features due to this. \\

	      The two perspectives of TD, from development to business, is different.
	      Developers are mainly interested in the perfection of the code whilst management considers that time to market is of extreme importance.
	      However, developers are more aware of the impact of TD over time, since they have to deal with it on a daily basis. \\

	      As a result of the study, the authors, developers and management have proposed the following practices when dealing with TD:
	      \begin{itemize}
		      \item Allocate 5-10\% of resources for each release in paying back TD.
		      \item Always keep an open dialogue with the customers and management on issues surrounding TD.
		      \item Make TD as visible as possible through the use of documentation, static analysis tools, etc.
	      \end{itemize}
	\item Links to other resources? \\
\end{itemize}

\subsubsection{Searching for Build Debt} \cite{Morgenthaler2012}
\begin{itemize}
	\item What were the goals of the paper? \\
	      To publish enterprise approaches within Google for managing technical debt in build files.
	      This type of TD was lowering developer productivity and increased the computational resources within the infrastructure and thus increased costs.
	\item What was the methodology? \\
	      The developers at Google leveraged in-door developed tools for automating removal of legacy code projects no longer refenced, removal of indirect dependencies between projects and removal of dead command-line flags which hindered developer comprehension of command line tools and processes.
	      Teams were holding FIXIT days where developers may remove and improve dead flags and dead/zombie dependencies.
	\item What were the conclusions? \\
	      Google has a very large codebase that is difficult to manage.
	      Specifically, a simple dependency change might take weeks due to previous links in the dependency graphs and build files of other projects that rely on it.
	      Removal of dead code has been simplied with the introduction of code visibility and health flags.
	      Additionally, one FIXIT day has removed approximately 250k lines of code related to dead flags.
	\item What were the limitations? \\
	      There was only one type of TD assessed in the paper: BUILD debt.
\end{itemize}

\subsubsection{Managing Technical Debt - An industrial case study} \cite{Codabux2013}
\begin{itemize}
	\item What were the goals of the paper? \\
	      The goals of the paper was to complete an empirical analysis of Agile methodologies and its effect on technical debt within a mid-sized company.
	      In particular, the authors wanted to find what were the various types of TD, how to handle them, what the consequences of holding on to that debt are, how TD is prioritized and addressed.
	\item What was the methodology? \\
	      The authors had 3 visits to the office of the company, each 3-day long.
	      They were part of team meetings such as sprint planning, scrum and retrospectives.
	      Interviews with both management and development teams have been completed.
	      A total of 28 members had participated to the study.
	\item What were the conclusions? \\
	      The authors had mixed responses when comparing division management interviews and developers.
	      Division management considered technical debt to be of infrastructure nature (in terms of operations) and testing nature (issues with test automation).
	      Developers, on the other hand, correlated TD with the lack of time to implement features \textit{properly}.\\
	      However, architecture debt was considered to be the most difficult to manage since it required group decisions which take time and coordination.
	      The team was taking appropriate steps to exterminate TD by assigning small teams within the division solely dedicated to refactoring tasks.
	\item What were the limitations? \\
	      The study was conducted with one industry partner and therefore does not reflect the impact of TD within the entire Agile process.
	      Additionally, interviews were not recorded and responses may be subject to bias of the interviewer.
	\item Links to other resources? \\
\end{itemize}

\subsubsection{Technical Debt and its Effects on Agile Management Practice} \cite{Holvitie2014}
\begin{itemize}
	\item What were the goals of the paper? \\
	      To study technical debt within the agile management process and its impact.
	\item What was the methodology? \\
	      The study consisted of three sections of research questions which covered background information and TD knowledge level, agile processes affecting TD and singular instances of TD.
	      Data collection was completed through an anonymous online form sent to 507 individuals spanning more than 150 companies.
	      The survey consisted of 34 questions open and closed-ended and took approximately 10 minutes to complete.
	\item What were the conclusions? \\
	      80\% of the individuals have heard of technical debt before and considered to have a good knowledge on it.
	      Of their entire day, TD was discussed most in work meetings while only 35\% of the responses have applied refactorings related to TD.\\
	      50\% of the participants considered practices such as simple design, TDD, following code standards, continous integration, collective ownership and pair programming to have a positive effect on TD.
	      Additionally, of all the meetings, retrospectives have been considered as most influential to TD by 50\% of the responses.
	      However, phases such as requirements gathering, design and implementation had been considered to incur the most TD.
	\item What were the limitations? \\
	      The study was conducted online anonymously - no way to understand how many sectors, how many companies and teams have been involved in the responses.\\
	      Limited to the country of the authors - Finland.
\end{itemize}

\subsubsection{An Enterprise Perspective on Technical Debt} \cite{Klinger2011}

\begin{itemize}
	\item What were the goals of the paper? \\
	      To challenge Ward Cunningham's definition of technical debt from an enterprise perspective and define it as a strategic tool for business circumstances.
	\item What was the methodology? \\
	      Interviewed four technical architects from IBM, from a "wide range" of projects within the firm. Each interview was approx. 1 hour in length.
	      The questions tapped into their own experiences with TD, more specifically: nature and context, stakeholders, benefits, documentation, whether the debt was repaid.
	\item What did the authors learn? \\
	      Enterprises may use debt as a tool for maximizing competitive advantage, which may or may not be paid in the future dependant on the direction in which the project will take.
	      Additionally, many stakeholders are involved in the accrual of debt in a software project (decisions are made collectively, not in a vacuum) but sometimes technical decisions would be made ad-hoc with no definite formal process.
	      However, the challenge is in the collective - it is difficult to understand the socio-technical background of a system. \\

	      Some stakeholders do not comprehend the extent of technical debt they are incurring in the system through their decisions due to:
	      \begin{itemize}
		      \item No vocabulary between technical architects and non-technical stakeholders.
		      \item Stakeholders having competing goals and "win" conditions which may be "suboptimal from a technical perspective".
	      \end{itemize}
	\item Links to other resources?
\end{itemize}

%%%%%%%%%%%%%%%%%%%%%%%%%%%%%%%%%%%%%%%%%%%%%%%%%%%%%%%%%%%%%%%%%%%
\section{Proposed Approach}

Goal-Question-Metric approach is a good framework for breaking down research work. (TODO: add citation here)

Goal-Question-Metric approach:
\begin{itemize}
	\item Goal: \textbf{Analyze} project feature implementations \textbf{with the purpose of} identifying extra work \textbf{from the viewpoint of} software engineers and project manager \textbf{in the context of} technical debt management.
	\item RQ1: Can technical debt interest for a team be predicted for each sprint?
	\item RQ1.1: What was the difference (delta) in the estimated work effort for a feature and the practical work effort due to technical debt?
	\item RQ1.2: What is the measurement of TD in the affected modules across change sets?
	\item RQ1.3: How does work effort delta vary with the magnitude of technical debt?
	\item RQ2: What are the development patterns surrounding feature lifecycle?
	\item RQ2.2: At what checkpoint in development is technical debt reduction (refactoring)  most prominent?
\end{itemize}

Possible approach:
1) Identify appropriate data candidates for this study. These are a mixture of open source and enterprise software.
2) Identify suitable work items from project planning software.
3) Retrieve work items from issue tracker.
4) Identify checkpoints in project codebase which are appropriate to the selected work items.
5) Analyze work effort of selected work items.
6) Analyze measurement of technical debt in change sets.
7) Analysis and discussion of results.

Risks:
\begin{enumerate}
	\item Data candidates may not contain enough relevant information in project management software, such as priority and story points.
	\item Work effort estimation is difficult in the context of open source systems. What is the metric for quantifying work effort?
	      \begin{itemize}
		      \item Time - theoretically used in Agile process through metrics such as story points.
		            Must be careful on how it is quantified since there may be many outliers.
		      \item Change sets - the amount of lines of code added and removed of a work item.
		            Possible outliers are due to refactoring: remove, add, copy, etc.
		            This issue is further emphasized by diversity of refactoring tools available in modern IDEs.
	      \end{itemize}
	\item Technical debt is not a standardized metric and various tools calculate TD with a different formula and take into account multiple types of debt.
	\item Granularity of code checkpoints is important. These may be: commit level, branch level, pull request level, release level.
\end{enumerate}

%%%%%%%%%%%%%%%%%%%%%%%%%%%%%%%%%%%%%%%%%%%%%%%%%%%%%%%%%%%%%%%%%%%
\section{Work Plan}

show how you plan to organize your work, identifying intermediate deliverables and dates.\\
\textbf{TBD}

%%%%%%%%%%%%%%%%%%%%%%%%%%%%%%%%%%%%%%%%%%%%%%%%%%%%%%%%%%%%%%%%%%%
\section{Conclusion}

Does it clearly explain the problem?
Does it contain a bibliography and proper citations?

Report: Is the report complete, well-organised, clear, and literate?
Overall: What is your overall impression of the student’s work?

%%%%%%%%%%%%%%%%%%%%%%%%%%%%%%%%%%%%%%%%%%%%%%%%%%%%%%%%%%%%%%%%%%%
% it is fine to change the bibliography style if you want
\bibliographystyle{plain}
\bibliography{mprop}
\end{document}
