\documentclass{mprop}
\usepackage{graphicx}

% alternative font if you prefer
%\usepackage{times}

% for alternative page numbering use the following package
% and see documentation for commands
%\usepackage{fancyheadings}


% other potentially useful packages
%\uspackage{amssymb,amsmath}
%\usepackage{url}
%\usepackage{fancyvrb}
%\usepackage[final]{pdfpages}

\begin{document}

%%%%%%%%%%%%%%%%%%%%%%%%%%%%%%%%%%%%%%%%%%%%%%%%%%%%%%%%%%%%%%%%%%%
\title{Technical Debt Title Here}
\author{Ovidiu Popoviciu}
\date{18th December 2018}
\maketitle
%%%%%%%%%%%%%%%%%%%%%%%%%%%%%%%%%%%%%%%%%%%%%%%%%%%%%%%%%%%%%%%%%%%

%%%%%%%%%%%%%%%%%%%%%%%%%%%%%%%%%%%%%%%%%%%%%%%%%%%%%%%%%%%%%%%%%%%
\tableofcontents
\newpage
%%%%%%%%%%%%%%%%%%%%%%%%%%%%%%%%%%%%%%%%%%%%%%%%%%%%%%%%%%%%%%%%%%%

%%%%%%%%%%%%%%%%%%%%%%%%%%%%%%%%%%%%%%%%%%%%%%%%%%%%%%%%%%%%%%%%%%%
\section{Introduction}\label{intro}

briefly explain the context of the project problem

Problem statement: Has the student analysed the problem, stated it clearly, and justified its importance?

\subsection{A subsection}
Please note your proposal need not follow the included section headings - this is only a suggested structure. Also add subsections etc as required

example references: \cite{BK08}

%%%%%%%%%%%%%%%%%%%%%%%%%%%%%%%%%%%%%%%%%%%%%%%%%%%%%%%%%%%%%%%%%%%
\section{Statement of Problem}

clearly state the problem to be addressed in your forthcoming project. Explain why it would be worthwhile to solve this problem.

%%%%%%%%%%%%%%%%%%%%%%%%%%%%%%%%%%%%%%%%%%%%%%%%%%%%%%%%%%%%%%%%%%%
\section{Background Survey}

present an overview of relevant previous work including articles, books, and existing software products. Critically evaluate the strengths and weaknesses of the previous work.

Survey: Has the student surveyed and critically reviewed relevant previous work, including literature and software products?

\subsection{The SQALE Method for Measuring Technical Debt} \cite{Letouzey2012}
\begin{itemize}
	\item What were the goals of the paper? \\
		The goal of this paper was to propose a standardized, language-agnostic framework for assessing the quality of source code by deriving measures for important code characteristics and calculating technical debt.
	\item What was the methodology? \\
		The framework proposed consisted of four concepts:
		\begin{itemize}
			\item Quality Model - defines internal properties of code through a structured three-layer hierarchy (characteristic, sub-characteristic and requirement).
			\item Analysis Model - measurement of the distance between the current state of the application and the \"optimized\" state, the quality target.
			\item Indices - each characteristic of the Quality Model defines a remediation index (cost to repair the non-compliances) defined in time, work or capital units.
			\item Indicators - provide a visual representation of technical debt either through ratings (ratios between TD and development cost) or SQALE Pyramids (distribution of TD over all the characteristics).
		\end{itemize}
	\item What did the authors learn?
		\begin{itemize}
			\item Each organization must manage technical debt as early as possible in a project with indicators and dashboards for code quality available for each build within the continous integration pipeline.
			\item Quality model must be calibrated according to the requirements and policies of the organization implementing the framework.
			\item One limitation is that the framework does not take into consideration other types of technical debt such as requirements debt, operational debt.
			\item There is no standardized definition of right code, organizations have to define their own \"right\" code rules.
		\end{itemize}
	\item Links to other resources?
		-
\end{itemize}
%%%%%%%%%%%%%%%%%%%%%%%%%%%%%%%%%%%%%%%%%%%%%%%%%%%%%%%%%%%%%%%%%%%
\section{Proposed Approach}

state how you propose to solve the software development problem. Show that your proposed approach is feasible, but identify any risks.

%%%%%%%%%%%%%%%%%%%%%%%%%%%%%%%%%%%%%%%%%%%%%%%%%%%%%%%%%%%%%%%%%%%
\section{Work Plan}

show how you plan to organize your work, identifying intermediate deliverables and dates.

%%%%%%%%%%%%%%%%%%%%%%%%%%%%%%%%%%%%%%%%%%%%%%%%%%%%%%%%%%%%%%%%%%%
\section{Conclusion}

Report: Is the report complete, well-organised, clear, and literate? Does it clearly explain the problem? Does it contain a bibliography and proper citations?
Overall: What is your overall impression of the student’s work?

%%%%%%%%%%%%%%%%%%%%%%%%%%%%%%%%%%%%%%%%%%%%%%%%%%%%%%%%%%%%%%%%%%%
% it is fine to change the bibliography style if you want
\bibliographystyle{plain}
\bibliography{mprop}
\end{document}
